\documentclass[journal,12pt,twocolumn]{IEEEtran}

\usepackage{setspace}
\usepackage{gensymb}

\singlespacing


\usepackage[cmex10]{amsmath}

\usepackage{amsthm}

\usepackage{mathrsfs}
\usepackage{txfonts}
\usepackage{stfloats}
\usepackage{bm}
\usepackage{cite}
\usepackage{cases}
\usepackage{subfig}

\usepackage{longtable}
\usepackage{multirow}

\usepackage{enumitem}
\usepackage{mathtools}
\usepackage{steinmetz}
\usepackage{tikz}
\usepackage{circuitikz}
\usepackage{verbatim}
\usepackage{tfrupee}
\usepackage[breaklinks=true]{hyperref}
\usepackage{graphicx}
\usepackage{tkz-euclide}

\usetikzlibrary{calc,math}
\usepackage{listings}
    \usepackage{color}                                            %%
    \usepackage{array}                                            %%
    \usepackage{longtable}                                        %%
    \usepackage{calc}                                             %%
    \usepackage{multirow}                                         %%
    \usepackage{hhline}                                           %%
    \usepackage{ifthen}                                           %%
    \usepackage{lscape}     
\usepackage{multicol}
\usepackage{chngcntr}

\DeclareMathOperator*{\Res}{Res}

\renewcommand\thesection{\arabic{section}}
\renewcommand\thesubsection{\thesection.\arabic{subsection}}
\renewcommand\thesubsubsection{\thesubsection.\arabic{subsubsection}}

\renewcommand\thesectiondis{\arabic{section}}
\renewcommand\thesubsectiondis{\thesectiondis.\arabic{subsection}}
\renewcommand\thesubsubsectiondis{\thesubsectiondis.\arabic{subsubsection}}


\hyphenation{op-tical net-works semi-conduc-tor}
\def\inputGnumericTable{}                                 %%

\lstset{
%language=C,
frame=single, 
breaklines=true,
columns=fullflexible
}
\begin{document}


\newtheorem{theorem}{Theorem}[section]
\newtheorem{problem}{Problem}
\newtheorem{proposition}{Proposition}[section]
\newtheorem{lemma}{Lemma}[section]
\newtheorem{corollary}[theorem]{Corollary}
\newtheorem{example}{Example}[section]
\newtheorem{definition}[problem]{Definition}

\newcommand{\BEQA}{\begin{eqnarray}}
\newcommand{\EEQA}{\end{eqnarray}}
\newcommand{\define}{\stackrel{\triangle}{=}}
\bibliographystyle{IEEEtran}
\providecommand{\mbf}{\mathbf}
\providecommand{\pr}[1]{\ensuremath{\Pr\left(#1\right)}}
\providecommand{\qfunc}[1]{\ensuremath{Q\left(#1\right)}}
\providecommand{\sbrak}[1]{\ensuremath{{}\left[#1\right]}}
\providecommand{\lsbrak}[1]{\ensuremath{{}\left[#1\right.}}
\providecommand{\rsbrak}[1]{\ensuremath{{}\left.#1\right]}}
\providecommand{\brak}[1]{\ensuremath{\left(#1\right)}}
\providecommand{\lbrak}[1]{\ensuremath{\left(#1\right.}}
\providecommand{\rbrak}[1]{\ensuremath{\left.#1\right)}}
\providecommand{\cbrak}[1]{\ensuremath{\left\{#1\right\}}}
\providecommand{\lcbrak}[1]{\ensuremath{\left\{#1\right.}}
\providecommand{\rcbrak}[1]{\ensuremath{\left.#1\right\}}}
\theoremstyle{remark}
\newtheorem{rem}{Remark}
\newcommand{\sgn}{\mathop{\mathrm{sgn}}}
\providecommand{\abs}[1]{\vert#1\vert}
\providecommand{\res}[1]{\Res\displaylimits_{#1}} 
\providecommand{\norm}[1]{\Vert#1\rVert}
%\providecommand{\norm}[1]{\lVert#1\rVert}
\providecommand{\mtx}[1]{\mathbf{#1}}
\providecommand{\mean}[1]{E[ #1 ]}
\providecommand{\fourier}{\overset{\mathcal{F}}{ \rightleftharpoons}}
%\providecommand{\hilbert}{\overset{\mathcal{H}}{ \rightleftharpoons}}
\providecommand{\system}{\overset{\mathcal{H}}{ \longleftrightarrow}}
	%\newcommand{\solution}[2]{\textbf{Solution:}{#1}}
\newcommand{\solution}{\noindent \textbf{Solution: }}
\newcommand{\cosec}{\,\text{cosec}\,}
\providecommand{\dec}[2]{\ensuremath{\overset{#1}{\underset{#2}{\gtrless}}}}
\newcommand{\myvec}[1]{\ensuremath{\begin{pmatrix}#1\end{pmatrix}}}
\newcommand{\mydet}[1]{\ensuremath{\begin{vmatrix}#1\end{vmatrix}}}
\numberwithin{equation}{subsection}
\makeatletter
\@addtoreset{figure}{problem}
\makeatother
\let\StandardTheFigure\thefigure
\let\vec\mathbf
\renewcommand{\thefigure}{\theproblem}
\def\putbox#1#2#3{\makebox[0in][l]{\makebox[#1][l]{}\raisebox{\baselineskip}[0in][0in]{\raisebox{#2}[0in][0in]{#3}}}}
     \def\rightbox#1{\makebox[0in][r]{#1}}
     \def\centbox#1{\makebox[0in]{#1}}
     \def\topbox#1{\raisebox{-\baselineskip}[0in][0in]{#1}}
     \def\midbox#1{\raisebox{-0.5\baselineskip}[0in][0in]{#1}}
\vspace{3cm}
\title{Assignment-2}
\author{Satya Sangram Mishra}
\maketitle
\newpage
\bigskip
\renewcommand{\thefigure}{\theenumi}
\renewcommand{\thetable}{\theenumi}
Download all python codes from 
\begin{lstlisting}
https://github.com/satyasm45/Summer-Internship/tree/main/Assignment-2/Codes
\end{lstlisting}
%
and latex-tikz codes from 
%
\begin{lstlisting}
https://github.com/satyasm45/Summer-Internship/tree/main/Assignment-2/Figs
\end{lstlisting}
%
\section{Question No. 2.28}
Construct a quadrilateral ABCD such that AB =5, \angle A =50$^\circ$
, AC = 4, BD = 5 and AD = 6.
%
\section{Explanation}
The rough figure of the expected quadrilateral ABCD is given in Figure 2.1.

\numberwithin{figure}{section}
\begin{figure}[!ht]
\centering
\resizebox{!}{!}{\begin{tikzpicture}
[scale=1.5,>=stealth,point/.style={draw,circle,fill = black,inner sep=0.5pt},]



%Coordinates of D

\def\p{3.6}
\def\q{{4.8}}

%Labeling points
\node (D) at (\p,\q)[point,label=above right:$D$] {};
\node (A) at (0, 0)[point,label=below left:$A$] {};
\node (B) at (5, 0)[point,label=below right:$B$] {};


%Coordinates of C

\node (C) at (3,2.64)[point,label=above right:$C$] {};


%Drawing quadrilateral ABCD
\draw (A) -- (B) --  (C) --(D)--(A);
\draw (A) -- (C);
\draw (B) --(D);


\tkzMarkAngle[fill=green!20](B,A,D)
\draw (1,0.5) node [anchor=north west][inner sep=0.75pt]   [align=left] {$50^o$};

\draw (3,-0.2) node [anchor=north west][inner sep=0.75pt]   [align=left] {$5$};

\draw (2,3.2) node [anchor=north west][inner sep=0.75pt]   [align=left] {$6$};

\draw (4.2,3.2) node [anchor=north west][inner sep=0.75pt]   [align=left] {$5$};

\draw (2,1.5) node [anchor=north west][inner sep=0.75pt]   [align=left] {$4$};

\end{tikzpicture}

}
\caption{Rough Figure}
\label{fig:myfigure}	
\end{figure}

For this quadrilateral adjacent side lengths AB,AD and diagonal BD is known. Three sides of $\triangle ABD$ are therefore known. 

So, $\angle{A}$ can also be found out using the Cosine Rule. But value for $\angle {A}$ is given. So we need to verify it.

\begin{align}
\cos{A}=\frac{ (\norm{\vec{B}-\vec{A}})^2+ (\norm{\vec{D}-\vec{A}})^2- (\norm{\vec{D}-\vec{B}})^2}{2\times(\norm{\vec{B}-\vec{A}})\times (\norm{\vec{D}-\vec{A}})}
\end{align}
\begin{align}
\cos{A}=\frac{5^2+6^2-5^2}{2\times5\times6}
\\
\implies \angle A=\cos^{-1}(0.6)
\end{align}
 
 So \angle{A}=53.13$^\circ$.

 But \angle{A}=50$^\circ$\ is given which causes a mismatch. 

Therefore construction of quadrilateral with given measurements is not possible.

\end{document}